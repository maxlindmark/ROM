\documentclass[]{article}
\usepackage{lmodern}
\usepackage{amssymb,amsmath}
\usepackage{ifxetex,ifluatex}
\usepackage{fixltx2e} % provides \textsubscript
\ifnum 0\ifxetex 1\fi\ifluatex 1\fi=0 % if pdftex
  \usepackage[T1]{fontenc}
  \usepackage[utf8]{inputenc}
\else % if luatex or xelatex
  \ifxetex
    \usepackage{mathspec}
  \else
    \usepackage{fontspec}
  \fi
  \defaultfontfeatures{Ligatures=TeX,Scale=MatchLowercase}
\fi
% use upquote if available, for straight quotes in verbatim environments
\IfFileExists{upquote.sty}{\usepackage{upquote}}{}
% use microtype if available
\IfFileExists{microtype.sty}{%
\usepackage{microtype}
\UseMicrotypeSet[protrusion]{basicmath} % disable protrusion for tt fonts
}{}
\usepackage[margin=1in]{geometry}
\usepackage{hyperref}
\hypersetup{unicode=true,
            pdftitle={Making figures for RoM in R},
            pdfauthor={Max Lindmark},
            pdfborder={0 0 0},
            breaklinks=true}
\urlstyle{same}  % don't use monospace font for urls
\usepackage{color}
\usepackage{fancyvrb}
\newcommand{\VerbBar}{|}
\newcommand{\VERB}{\Verb[commandchars=\\\{\}]}
\DefineVerbatimEnvironment{Highlighting}{Verbatim}{commandchars=\\\{\}}
% Add ',fontsize=\small' for more characters per line
\usepackage{framed}
\definecolor{shadecolor}{RGB}{248,248,248}
\newenvironment{Shaded}{\begin{snugshade}}{\end{snugshade}}
\newcommand{\KeywordTok}[1]{\textcolor[rgb]{0.13,0.29,0.53}{\textbf{#1}}}
\newcommand{\DataTypeTok}[1]{\textcolor[rgb]{0.13,0.29,0.53}{#1}}
\newcommand{\DecValTok}[1]{\textcolor[rgb]{0.00,0.00,0.81}{#1}}
\newcommand{\BaseNTok}[1]{\textcolor[rgb]{0.00,0.00,0.81}{#1}}
\newcommand{\FloatTok}[1]{\textcolor[rgb]{0.00,0.00,0.81}{#1}}
\newcommand{\ConstantTok}[1]{\textcolor[rgb]{0.00,0.00,0.00}{#1}}
\newcommand{\CharTok}[1]{\textcolor[rgb]{0.31,0.60,0.02}{#1}}
\newcommand{\SpecialCharTok}[1]{\textcolor[rgb]{0.00,0.00,0.00}{#1}}
\newcommand{\StringTok}[1]{\textcolor[rgb]{0.31,0.60,0.02}{#1}}
\newcommand{\VerbatimStringTok}[1]{\textcolor[rgb]{0.31,0.60,0.02}{#1}}
\newcommand{\SpecialStringTok}[1]{\textcolor[rgb]{0.31,0.60,0.02}{#1}}
\newcommand{\ImportTok}[1]{#1}
\newcommand{\CommentTok}[1]{\textcolor[rgb]{0.56,0.35,0.01}{\textit{#1}}}
\newcommand{\DocumentationTok}[1]{\textcolor[rgb]{0.56,0.35,0.01}{\textbf{\textit{#1}}}}
\newcommand{\AnnotationTok}[1]{\textcolor[rgb]{0.56,0.35,0.01}{\textbf{\textit{#1}}}}
\newcommand{\CommentVarTok}[1]{\textcolor[rgb]{0.56,0.35,0.01}{\textbf{\textit{#1}}}}
\newcommand{\OtherTok}[1]{\textcolor[rgb]{0.56,0.35,0.01}{#1}}
\newcommand{\FunctionTok}[1]{\textcolor[rgb]{0.00,0.00,0.00}{#1}}
\newcommand{\VariableTok}[1]{\textcolor[rgb]{0.00,0.00,0.00}{#1}}
\newcommand{\ControlFlowTok}[1]{\textcolor[rgb]{0.13,0.29,0.53}{\textbf{#1}}}
\newcommand{\OperatorTok}[1]{\textcolor[rgb]{0.81,0.36,0.00}{\textbf{#1}}}
\newcommand{\BuiltInTok}[1]{#1}
\newcommand{\ExtensionTok}[1]{#1}
\newcommand{\PreprocessorTok}[1]{\textcolor[rgb]{0.56,0.35,0.01}{\textit{#1}}}
\newcommand{\AttributeTok}[1]{\textcolor[rgb]{0.77,0.63,0.00}{#1}}
\newcommand{\RegionMarkerTok}[1]{#1}
\newcommand{\InformationTok}[1]{\textcolor[rgb]{0.56,0.35,0.01}{\textbf{\textit{#1}}}}
\newcommand{\WarningTok}[1]{\textcolor[rgb]{0.56,0.35,0.01}{\textbf{\textit{#1}}}}
\newcommand{\AlertTok}[1]{\textcolor[rgb]{0.94,0.16,0.16}{#1}}
\newcommand{\ErrorTok}[1]{\textcolor[rgb]{0.64,0.00,0.00}{\textbf{#1}}}
\newcommand{\NormalTok}[1]{#1}
\usepackage{graphicx,grffile}
\makeatletter
\def\maxwidth{\ifdim\Gin@nat@width>\linewidth\linewidth\else\Gin@nat@width\fi}
\def\maxheight{\ifdim\Gin@nat@height>\textheight\textheight\else\Gin@nat@height\fi}
\makeatother
% Scale images if necessary, so that they will not overflow the page
% margins by default, and it is still possible to overwrite the defaults
% using explicit options in \includegraphics[width, height, ...]{}
\setkeys{Gin}{width=\maxwidth,height=\maxheight,keepaspectratio}
\IfFileExists{parskip.sty}{%
\usepackage{parskip}
}{% else
\setlength{\parindent}{0pt}
\setlength{\parskip}{6pt plus 2pt minus 1pt}
}
\setlength{\emergencystretch}{3em}  % prevent overfull lines
\providecommand{\tightlist}{%
  \setlength{\itemsep}{0pt}\setlength{\parskip}{0pt}}
\setcounter{secnumdepth}{0}
% Redefines (sub)paragraphs to behave more like sections
\ifx\paragraph\undefined\else
\let\oldparagraph\paragraph
\renewcommand{\paragraph}[1]{\oldparagraph{#1}\mbox{}}
\fi
\ifx\subparagraph\undefined\else
\let\oldsubparagraph\subparagraph
\renewcommand{\subparagraph}[1]{\oldsubparagraph{#1}\mbox{}}
\fi

%%% Use protect on footnotes to avoid problems with footnotes in titles
\let\rmarkdownfootnote\footnote%
\def\footnote{\protect\rmarkdownfootnote}

%%% Change title format to be more compact
\usepackage{titling}

% Create subtitle command for use in maketitle
\providecommand{\subtitle}[1]{
  \posttitle{
    \begin{center}\large#1\end{center}
    }
}

\setlength{\droptitle}{-2em}

  \title{Making figures for RoM in R}
    \pretitle{\vspace{\droptitle}\centering\huge}
  \posttitle{\par}
    \author{Max Lindmark}
    \preauthor{\centering\large\emph}
  \postauthor{\par}
      \predate{\centering\large\emph}
  \postdate{\par}
    \date{28 april 2019}

\usepackage{booktabs} \usepackage{longtable} \usepackage{array}
\usepackage{multirow} \usepackage[table]{xcolor} \usepackage{wrapfig}
\usepackage{float} \floatplacement{figure}{H}

\begin{document}
\maketitle

\subsection{Introduction}\label{introduction}

\emph{First, this is a first draft and a work in progress so any
feedback on this document or the figures are highly appreciated!}

\subsubsection{Why R?}\label{why-r}

The main argument for using code to create RoM figures is to standardize
them across species and to do so while limiting repetitive work (you
only need to write a script once!). These figures here are made with the
package \emph{ggplot2}, which is a very powerful tool for making
multilevel figures and allows for easy modification via the
theme-function.

\subsubsection{Basic prerequisitis}\label{basic-prerequisitis}

It is an advantage if you know some basic R. I strongly recommend using
R-studio and working in a so called R-studio project (but it is not
needed to reproduce this code). To create a project, open R-studio,
click File/New Project/New Directory and specify where you want to save
it. Open the \emph{``your\_project\_name''.Rproj} and click File/New
Script and save that in your project folder. And that's it! The best
thing is that now all your search paths are relative and not absolute.
If you want to read in data, put the data inside the project folder and
you don't have to specify the full search path
(\emph{``C:/R/RoM/data-file.csv''} or whatever), it's enough you give
the name of the file only (more on that below!). Another benefit with a
relative search path is that I don't have to worry about setting the
working directory and changing the directory each time I try to rerun
anyone elses script.

Lastly, this is an R Markdown document. This means R-code is text with
grey background. You can copy these chunks of code to a new R-script in
your R-studio project and run it from there.

\subsubsection{What does this script
do?}\label{what-does-this-script-do}

This script gives an example of how you can use R to create standardized
figures in R, and save them in a given resolution as a \emph{.tiff},
ready to be used without further editing.

I have chosen data for freshwater pike (\emph{Esox lucious}) to
illustrate how you can use this code, because it has all potential data
for a RoM species (recreational, multiple areas, error bars etc.). These
data have been uploaded on github:
\href{https://github.com/maxlindmark/ROM}{github}

\subsubsection{How do I use this for my
species?}\label{how-do-i-use-this-for-my-species}

Most basic RoM figures that all species have in some form can be grouped
into three different categories. Each category will have its own script.
Once you identify the best category for your species you \textbf{should
not have to modify the actual plotting code}, instead the aim to is to

\begin{itemize}
\item
  \begin{enumerate}
  \def\labelenumi{\arabic{enumi}.}
  \tightlist
  \item
    pre-define your variables
  \end{enumerate}
\item
  \begin{enumerate}
  \def\labelenumi{\arabic{enumi}.}
  \setcounter{enumi}{1}
  \tightlist
  \item
    put hash tags in front of stuff you don't need! E.g. if you don't
    have points to plot, make sure to put a hash tag before the
    geom\_point()-function, which plots points. But do not worry!
    Instructions for how to do that will be in the described below! The
    categories are:
  \end{enumerate}
\item
  Fig. 1. Single series
\item
  Fig. 2. Multiple series
\item
  Fig. 3. Multiple series and y-axes - \textbf{\emph{not yet
  implemented, see end of document!}}
\end{itemize}

Check out the flowchart below to help identify which plot you need.

\begin{figure}
\centering
\includegraphics{Flow_chart.png}
\caption{How to select the best template for \textbf{your} species}
\end{figure}

Once you identify which plot type you will use, you can read about how
to prepare your data to use the script. A standardized way of saving
data is key for making any kind of standardized script! The data
structure is described in section \textbf{Preparation}.

\subsection{Preparation}\label{preparation}

\subsubsection{Load libraries}\label{load-libraries}

Before starting, we need to install a few packages:

\begin{Shaded}
\begin{Highlighting}[]
\KeywordTok{rm}\NormalTok{(}\DataTypeTok{list =} \KeywordTok{ls}\NormalTok{()) }\CommentTok{# clear the workspace from objects}

\CommentTok{# Provide package names}
\NormalTok{pkgs <-}\StringTok{ }\KeywordTok{c}\NormalTok{(}\StringTok{"devtools"}\NormalTok{, }
          \StringTok{"ggplot2"}\NormalTok{, }
          \StringTok{"RCurl"}\NormalTok{, }
          \StringTok{"RCurl"}\NormalTok{, }
          \StringTok{"tidyr"}\NormalTok{, }
          \StringTok{"dplyr"}\NormalTok{, }
          \StringTok{"scales"}\NormalTok{, }
          \StringTok{"png"}\NormalTok{, }
          \StringTok{"knitr"}\NormalTok{)}

\CommentTok{# Install packages}
\CommentTok{#install.packages(pkgs) # remove the hashtag if don't have them installed}

\CommentTok{# Load all packages}
\KeywordTok{invisible}\NormalTok{(}\KeywordTok{lapply}\NormalTok{(pkgs, }\ControlFlowTok{function}\NormalTok{(x) }\KeywordTok{require}\NormalTok{(x, }\DataTypeTok{character.only =}\NormalTok{ T, }\DataTypeTok{quietly =}\NormalTok{ T)))}
\end{Highlighting}
\end{Shaded}

\subsubsection{Load example data and clean it
up!}\label{load-example-data-and-clean-it-up}

Now let's read in the example data (freshwater pike):

\begin{Shaded}
\begin{Highlighting}[]
\CommentTok{# Go to https://github.com/maxlindmark/ROM to view the data in the browser}
\NormalTok{dat <-}\StringTok{ }\KeywordTok{read.csv}\NormalTok{(}
  \DataTypeTok{text =} \KeywordTok{getURL}\NormalTok{(}\StringTok{"https://raw.githubusercontent.com/maxlindmark/ROM/master/pike.csv"}\NormalTok{), }
  \DataTypeTok{sep =} \StringTok{";"}\NormalTok{)}
\end{Highlighting}
\end{Shaded}

For your own species, the code would typically look like this if you put
the data inside your R Project folder:

\begin{Shaded}
\begin{Highlighting}[]
\NormalTok{dat <-}\StringTok{ }\KeywordTok{read.csv}\NormalTok{(}\StringTok{"pike.csv"}\NormalTok{, }\DataTypeTok{sep =} \StringTok{";"}\NormalTok{)}
\end{Highlighting}
\end{Shaded}

Inspect the data:

\begin{Shaded}
\begin{Highlighting}[]
\KeywordTok{head}\NormalTok{(dat)}
\end{Highlighting}
\end{Shaded}

\begin{verbatim}
##   X.c5.r error rec_plus rec_minu        Omr.e5.de Biomassa
## 1   1997    NA       NA       NA Stora sj<f6>arna      115
## 2   1998    NA       NA       NA Stora sj<f6>arna      114
## 3   1999    NA       NA       NA Stora sj<f6>arna      149
## 4   2000    NA       NA       NA Stora sj<f6>arna      145
## 5   2001    NA       NA       NA Stora sj<f6>arna      121
## 6   2002    NA       NA       NA Stora sj<f6>arna      145
\end{verbatim}

Users will have different encoding which affects how certain characters
are read. Instead of forcing everyone to use the same encoding, I'm
showing you how to rename columns. It is \textbf{VERY} important that
you use these column names because they will determine which level gets
which colour and line (if you insist on other column names you have to
modify the plot code)!

\begin{Shaded}
\begin{Highlighting}[]
\NormalTok{dat <-}\StringTok{ }\NormalTok{dat }\OperatorTok\StringTok{ }\KeywordTok{rename}\NormalTok{(Områ}\DataTypeTok{de =}\NormalTok{ Omr.e5.de,}
\NormalTok{                      Å}\DataTypeTok{r     =}\NormalTok{ X.c5.r)}
\KeywordTok{head}\NormalTok{(dat)}
\end{Highlighting}
\end{Shaded}

\begin{verbatim}
##     År error rec_plus rec_minu           Område Biomassa
## 1 1997    NA       NA       NA Stora sj<f6>arna      115
## 2 1998    NA       NA       NA Stora sj<f6>arna      114
## 3 1999    NA       NA       NA Stora sj<f6>arna      149
## 4 2000    NA       NA       NA Stora sj<f6>arna      145
## 5 2001    NA       NA       NA Stora sj<f6>arna      121
## 6 2002    NA       NA       NA Stora sj<f6>arna      145
\end{verbatim}

Columns look better! But we also need to modify the actual levels of the
column (i.e.~the levels of the lakes). You can see which lakes are in
the data by typing:

\begin{Shaded}
\begin{Highlighting}[]
\KeywordTok{levels}\NormalTok{(dat}\OperatorTok{$}\NormalTok{Område)}
\end{Highlighting}
\end{Shaded}

\begin{verbatim}
## [1] "Fritidsfiske"     "Hj<e4>lmaren"     "M<e4>laren"      
## [4] "Stora sj<f6>arna" "V<e4>nern"        "V<e4>ttern"
\end{verbatim}

Clearly we also need to correct the names of the levels, similar to how
we did for column names. This is because the beauty of \emph{ggplot} is
that all important information in the plot will be inherited from the
data.

\begin{Shaded}
\begin{Highlighting}[]
\KeywordTok{levels}\NormalTok{(dat}\OperatorTok{$}\NormalTok{Område) <-}\StringTok{ }\KeywordTok{c}\NormalTok{(}\StringTok{"Fritidsfiske"}\NormalTok{, }
                        \StringTok{"Hjälmaren"}\NormalTok{, }
                        \StringTok{"Mälaren"}\NormalTok{, }
                        \StringTok{"Stora sjöarna"}\NormalTok{, }
                        \StringTok{"Vänern"}\NormalTok{, }
                        \StringTok{"Vättern"}\NormalTok{)}
\KeywordTok{head}\NormalTok{(dat)}
\end{Highlighting}
\end{Shaded}

\begin{verbatim}
##     År error rec_plus rec_minu        Område Biomassa
## 1 1997    NA       NA       NA Stora sjöarna      115
## 2 1998    NA       NA       NA Stora sjöarna      114
## 3 1999    NA       NA       NA Stora sjöarna      149
## 4 2000    NA       NA       NA Stora sjöarna      145
## 5 2001    NA       NA       NA Stora sjöarna      121
## 6 2002    NA       NA       NA Stora sjöarna      145
\end{verbatim}

There is one last thing you might need to do with the data before
proceeding with plotting, and that is to change the order of the levels
in the data. When plotting, \emph{ggplot} sets the levels of the data in
alphabetical order, but here we want a specific order: the first
``level'' should correspond to the total and the last to any special
level, such as the recreational data.

\begin{Shaded}
\begin{Highlighting}[]
\NormalTok{dat}\OperatorTok{$}\NormalTok{Område <-}\StringTok{ }\KeywordTok{factor}\NormalTok{(dat}\OperatorTok{$}\NormalTok{Område,}
                     \DataTypeTok{levels =} \KeywordTok{c}\NormalTok{(}\StringTok{"Stora sjöarna"}\NormalTok{, }
                                \StringTok{"Vänern"}\NormalTok{, }
                                \StringTok{"Vättern"}\NormalTok{, }
                                \StringTok{"Mälaren"}\NormalTok{, }
                                \StringTok{"Hjälmaren"}\NormalTok{, }
                                \StringTok{"Fritidsfiske"}\NormalTok{))}
\end{Highlighting}
\end{Shaded}

Now the data look much better! You might not need to do these
modification on your own data. \textbf{But you need} to make sure it is
structured in the same way, that is: \textbf{1 row = 1 observation (not
multiple columns for different areas)}

\subsubsection{\texorpdfstring{Define \emph{ggplot}
theme}{Define ggplot theme}}\label{define-ggplot-theme}

Now that you have loaded and cleaned up your or the example data, we can
move on with general plotting settings. First the color palette:

\begin{Shaded}
\begin{Highlighting}[]
\NormalTok{pal <-}\StringTok{ }\KeywordTok{c}\NormalTok{(}\StringTok{"#56B4E9"}\NormalTok{, }\StringTok{"#009E73"}\NormalTok{, }\StringTok{"#F0E442"}\NormalTok{, }\StringTok{"#0072B2"}\NormalTok{, }\StringTok{"#E69F00"}\NormalTok{, }\StringTok{"#D55E00"}\NormalTok{)}
\end{Highlighting}
\end{Shaded}

Second, we define the theme we will use for all plots. This is made to
match as closely as possible to the RoM style. This applies to all
figures styles. In the future this function should be sourced but here I
show it in full. Make sure you copy paste this function and run it.

\begin{Shaded}
\begin{Highlighting}[]
\NormalTok{theme_rom <-}\StringTok{ }\ControlFlowTok{function}\NormalTok{(}\DataTypeTok{base_size =} \DecValTok{12}\NormalTok{, }\DataTypeTok{base_family =} \StringTok{""}\NormalTok{) \{}
  \KeywordTok{theme_bw}\NormalTok{(}\DataTypeTok{base_size =} \DecValTok{12}\NormalTok{, }\DataTypeTok{base_family =} \StringTok{""}\NormalTok{) }\OperatorTok{+}
\StringTok{    }\KeywordTok{theme}\NormalTok{(}
      \DataTypeTok{axis.text =} \KeywordTok{element_text}\NormalTok{(}\DataTypeTok{size =} \DecValTok{8}\NormalTok{), }
      \DataTypeTok{axis.title =} \KeywordTok{element_text}\NormalTok{(}\DataTypeTok{size =} \DecValTok{8}\NormalTok{),}
      \DataTypeTok{axis.ticks.length =} \KeywordTok{unit}\NormalTok{(}\FloatTok{0.05}\NormalTok{, }\StringTok{"cm"}\NormalTok{),}
      \DataTypeTok{axis.line =} \KeywordTok{element_line}\NormalTok{(}\DataTypeTok{colour =} \StringTok{"black"}\NormalTok{,}
                               \DataTypeTok{size =} \FloatTok{0.3}\NormalTok{), }
      \DataTypeTok{text =} \KeywordTok{element_text}\NormalTok{(}\DataTypeTok{family =} \StringTok{"sans"}\NormalTok{),}
      \DataTypeTok{panel.grid.major =} \KeywordTok{element_blank}\NormalTok{(),}
      \DataTypeTok{panel.grid.minor =} \KeywordTok{element_blank}\NormalTok{(),}
      \DataTypeTok{panel.border =} \KeywordTok{element_blank}\NormalTok{(),}
      \DataTypeTok{plot.title =} \KeywordTok{element_text}\NormalTok{(}\DataTypeTok{hjust =} \FloatTok{0.5}\NormalTok{, }
                                \DataTypeTok{margin =} \KeywordTok{margin}\NormalTok{(}\DataTypeTok{b =} \OperatorTok{-}\DecValTok{3}\NormalTok{), }
                                \DataTypeTok{size =} \FloatTok{9.6}\NormalTok{, }
                                \DataTypeTok{face =} \StringTok{"bold"}\NormalTok{),}
      \DataTypeTok{legend.position =} \KeywordTok{c}\NormalTok{(}\FloatTok{0.5}\NormalTok{, }\OperatorTok{-}\FloatTok{0.25}\NormalTok{),}
      \DataTypeTok{legend.text =} \KeywordTok{element_text}\NormalTok{(}\DataTypeTok{size =} \DecValTok{8}\NormalTok{),}
      \DataTypeTok{legend.justification =} \StringTok{"bottom"}\NormalTok{, }
      \DataTypeTok{legend.background =} \KeywordTok{element_rect}\NormalTok{(}\DataTypeTok{fill =} \StringTok{"transparent"}\NormalTok{), }
      \DataTypeTok{legend.key =} \KeywordTok{element_rect}\NormalTok{(}\DataTypeTok{fill =} \StringTok{"transparent"}\NormalTok{),}
      \DataTypeTok{aspect.ratio =} \DecValTok{1}\NormalTok{,}
      \DataTypeTok{plot.margin =} \KeywordTok{unit}\NormalTok{(}\KeywordTok{c}\NormalTok{(}\FloatTok{5.5}\NormalTok{, }\FloatTok{5.5}\NormalTok{, }\DecValTok{20}\NormalTok{, }\FloatTok{5.5}\NormalTok{), }
                         \StringTok{"points"}\NormalTok{)}
\NormalTok{      )}
\NormalTok{\}}
\end{Highlighting}
\end{Shaded}

\subsection{Let's start making
figures!}\label{lets-start-making-figures}

\subsection{Fig. 1. Single series}\label{fig.-1.-single-series}

\begin{figure}

{\centering \includegraphics{sikloja} 

}

\caption{Siklöja in Mälaren (RoM 2018) - example of a single-series plot}\label{fig:plot sikloja}
\end{figure}

For illustration purposes, we will now pretend our example pike data set
only is a single series, by filtering it to only contain the area
``Stora Sjöarna''. If you have a ``Fig.1 situation''``, your data might
look something like this, with one column for year, one for area and one
for tonnes.

\begin{Shaded}
\begin{Highlighting}[]
\NormalTok{dat1 <-}\StringTok{ }\NormalTok{dat }\OperatorTok\StringTok{ }
\StringTok{  }\KeywordTok{select}\NormalTok{(År, Biomassa, Område) }\OperatorTok\StringTok{ }
\StringTok{  }\KeywordTok{filter}\NormalTok{(Område }\OperatorTok{==}\StringTok{ "Stora sjöarna"}\NormalTok{)}

\KeywordTok{head}\NormalTok{(dat1)}
\end{Highlighting}
\end{Shaded}

\begin{verbatim}
##     År Biomassa        Område
## 1 1997      115 Stora sjöarna
## 2 1998      114 Stora sjöarna
## 3 1999      149 Stora sjöarna
## 4 2000      145 Stora sjöarna
## 5 2001      121 Stora sjöarna
## 6 2002      145 Stora sjöarna
\end{verbatim}

We now need to specify the \textbf{y-axis title} and \textbf{plot
title}. This is what is meant by predefining variables! Instead of
changing the plotting code we define what we want it to plot on the
axis, for instance**. For the pike data we can set them as:

\begin{Shaded}
\begin{Highlighting}[]
\NormalTok{y_axis <-}\StringTok{ }\KeywordTok{c}\NormalTok{(}\StringTok{"Landningar (ton)"}\NormalTok{)}
\NormalTok{title  <-}\StringTok{ }\KeywordTok{c}\NormalTok{(}\StringTok{"Landningar"}\NormalTok{)}
\end{Highlighting}
\end{Shaded}

Now go ahead and create the plot:

\begin{Shaded}
\begin{Highlighting}[]
\NormalTok{p1 <-}\StringTok{ }\KeywordTok{ggplot}\NormalTok{(dat1, }\KeywordTok{aes}\NormalTok{(År, Biomassa)) }\OperatorTok{+}
\StringTok{  }\KeywordTok{geom_bar}\NormalTok{(}\DataTypeTok{data =}\NormalTok{ dat1, }
           \KeywordTok{aes}\NormalTok{(}\DataTypeTok{x =}\NormalTok{ År, }\DataTypeTok{y =}\NormalTok{ Biomassa), }\DataTypeTok{stat =} \StringTok{"identity"}\NormalTok{, }\DataTypeTok{color =}\NormalTok{ pal[}\DecValTok{1}\NormalTok{], }\DataTypeTok{fill =}\NormalTok{ pal[}\DecValTok{1}\NormalTok{], }
           \DataTypeTok{width =} \FloatTok{0.6}\NormalTok{) }\OperatorTok{+}
\StringTok{  }\KeywordTok{labs}\NormalTok{(}\DataTypeTok{x =} \StringTok{""}\NormalTok{, }\DataTypeTok{y =}\NormalTok{ y_axis) }\OperatorTok{+}\StringTok{ }\CommentTok{# here's where the axis title is called}
\StringTok{  }\KeywordTok{ggtitle}\NormalTok{(title) }\OperatorTok{+}\StringTok{           }\CommentTok{# here's where the title is called}
\StringTok{  }\KeywordTok{guides}\NormalTok{(}\DataTypeTok{color  =} \OtherTok{FALSE}\NormalTok{) }\OperatorTok{+}
\StringTok{  }\KeywordTok{scale_x_continuous}\NormalTok{(}\DataTypeTok{expand =} \KeywordTok{c}\NormalTok{(}\DecValTok{0}\NormalTok{, }\DecValTok{0}\NormalTok{), }\DataTypeTok{breaks =}\NormalTok{ scales}\OperatorTok{::}\KeywordTok{pretty_breaks}\NormalTok{(}\DataTypeTok{n =} \DecValTok{6}\NormalTok{)) }\OperatorTok{+}
\StringTok{  }\KeywordTok{scale_y_continuous}\NormalTok{(}\DataTypeTok{expand =} \KeywordTok{c}\NormalTok{(}\DecValTok{0}\NormalTok{, }\DecValTok{0}\NormalTok{), }\DataTypeTok{breaks =}\NormalTok{ scales}\OperatorTok{::}\KeywordTok{pretty_breaks}\NormalTok{(}\DataTypeTok{n =} \DecValTok{5}\NormalTok{)) }\OperatorTok{+}
\StringTok{  }\KeywordTok{theme_rom}\NormalTok{()                }\CommentTok{# here we call our predefined theme}
\end{Highlighting}
\end{Shaded}

\begin{figure}

{\centering \includegraphics{Making_figures_for_RoM_in_R_files/figure-latex/plot 1-1} 

}

\caption{Gädda in the "Great Lakes" (RoM 2018) - example of a single-series plot using R}\label{fig:plot 1}
\end{figure}

Save the file (to your working directory)

\begin{Shaded}
\begin{Highlighting}[]
\KeywordTok{ggsave}\NormalTok{(}\StringTok{"Fig_1.tiff"}\NormalTok{, }\DataTypeTok{plot =}\NormalTok{ p1, }\DataTypeTok{dpi =} \DecValTok{300}\NormalTok{, }\DataTypeTok{width =} \DecValTok{8}\NormalTok{, }\DataTypeTok{height =} \DecValTok{8}\NormalTok{, }\DataTypeTok{units =} \StringTok{"cm"}\NormalTok{)}
\end{Highlighting}
\end{Shaded}

\subsection{Fig. 2. Multiple series}\label{fig.-2.-multiple-series}

\begin{figure}

{\centering \includegraphics{gadda} 

}

\caption{Gädda in the "Great Lakes" (RoM 2018) - example of a multiple-series plot}\label{fig:plot gadda}
\end{figure}

We now need to specify y-axis titles and plot title, and in addition we
need to define how many levels we have! For the pike data we can set
them as:

\begin{Shaded}
\begin{Highlighting}[]
\NormalTok{y_axis <-}\StringTok{ }\KeywordTok{c}\NormalTok{(}\StringTok{"Landningar (ton)"}\NormalTok{)}
\NormalTok{title <-}\StringTok{ }\KeywordTok{c}\NormalTok{(}\StringTok{"Landningar"}\NormalTok{) }
\NormalTok{main_series <-}\StringTok{ }\KeywordTok{c}\NormalTok{(}\StringTok{"Stora sjöarna"}\NormalTok{)}
\NormalTok{n_lev <-}\StringTok{ }\KeywordTok{length}\NormalTok{(}\KeywordTok{unique}\NormalTok{(dat}\OperatorTok{$}\NormalTok{Område)) }
\NormalTok{special_series <-}\StringTok{ }\KeywordTok{c}\NormalTok{(}\StringTok{"Fritidsfiske"}\NormalTok{)}
\end{Highlighting}
\end{Shaded}

Note that we in this example use multiple data series, and that one of
them is is recreational fisheries that in turn has error bars.
\textbf{These need to be in columns, so that we have a high (rec\_plus)
and a low (rec\_minu) column. Note also that they are NA when the Område
is not equal to recreational fisheries}

\begin{Shaded}
\begin{Highlighting}[]
\KeywordTok{head}\NormalTok{(dat) }
\end{Highlighting}
\end{Shaded}

\begin{verbatim}
##     År error rec_plus rec_minu        Område Biomassa
## 1 1997    NA       NA       NA Stora sjöarna      115
## 2 1998    NA       NA       NA Stora sjöarna      114
## 3 1999    NA       NA       NA Stora sjöarna      149
## 4 2000    NA       NA       NA Stora sjöarna      145
## 5 2001    NA       NA       NA Stora sjöarna      121
## 6 2002    NA       NA       NA Stora sjöarna      145
\end{verbatim}

\begin{Shaded}
\begin{Highlighting}[]
\KeywordTok{tail}\NormalTok{(dat)}
\end{Highlighting}
\end{Shaded}

\begin{verbatim}
##       År error rec_plus rec_minu       Område Biomassa
## 121 2012    NA       NA       NA Fritidsfiske       NA
## 122 2013    NA       NA       NA Fritidsfiske       NA
## 123 2014    73      223       77 Fritidsfiske      150
## 124 2015    51      184       82 Fritidsfiske      133
## 125 2016    NA       NA       NA Fritidsfiske       NA
## 126 2017    NA       NA       NA Fritidsfiske       NA
\end{verbatim}

And with that set, we can make the second figure, with multiple levels:

\begin{Shaded}
\begin{Highlighting}[]
\NormalTok{p2 <-}\StringTok{ }\KeywordTok{ggplot}\NormalTok{(dat, }\KeywordTok{aes}\NormalTok{(År, Biomassa, }\DataTypeTok{color =}\NormalTok{ Område)) }\OperatorTok{+}
\StringTok{  }\KeywordTok{geom_bar}\NormalTok{(}\DataTypeTok{data =} \KeywordTok{subset}\NormalTok{(dat, Område }\OperatorTok{==}\StringTok{ }\NormalTok{main_series), }
           \KeywordTok{aes}\NormalTok{(}\DataTypeTok{x =}\NormalTok{ År, }\DataTypeTok{y =}\NormalTok{ Biomassa), }\DataTypeTok{stat =} \StringTok{"identity"}\NormalTok{, }\DataTypeTok{color =}\NormalTok{ pal[}\DecValTok{1}\NormalTok{], }\DataTypeTok{fill =}\NormalTok{ pal[}\DecValTok{1}\NormalTok{], }
           \DataTypeTok{width =} \FloatTok{0.6}\NormalTok{) }\OperatorTok{+}
\StringTok{  }\KeywordTok{geom_line}\NormalTok{(}\DataTypeTok{data =}\NormalTok{ dat, }\KeywordTok{aes}\NormalTok{(År, Biomassa, }\DataTypeTok{color =}\NormalTok{ Område, }\DataTypeTok{alpha =}\NormalTok{ Område), }
            \DataTypeTok{size =} \DecValTok{1}\NormalTok{) }\OperatorTok{+}\StringTok{ }
\StringTok{  }\KeywordTok{geom_point}\NormalTok{(}\DataTypeTok{data =} \KeywordTok{subset}\NormalTok{(dat, Område }\OperatorTok{==}\StringTok{ }\NormalTok{special_series), }\CommentTok{# here we set our special series}
             \KeywordTok{aes}\NormalTok{(År, Biomassa, }\DataTypeTok{fill =}\NormalTok{ Område), }\DataTypeTok{size =} \DecValTok{2}\NormalTok{, }\DataTypeTok{color =}\NormalTok{ pal[}\KeywordTok{max}\NormalTok{(n_lev)]) }\OperatorTok{+}\StringTok{ }
\StringTok{  }\CommentTok{# above the number of level enters (max(n_lev))}
\StringTok{  }\KeywordTok{geom_errorbar}\NormalTok{(}\DataTypeTok{data =} \KeywordTok{subset}\NormalTok{(dat, Område }\OperatorTok{==}\StringTok{ }\NormalTok{special_series), }
                \KeywordTok{aes}\NormalTok{(}\DataTypeTok{x =}\NormalTok{ År, }\DataTypeTok{ymin =}\NormalTok{ rec_minu, }\DataTypeTok{ymax =}\NormalTok{ rec_plus), }
                \DataTypeTok{show.legend =} \OtherTok{FALSE}\NormalTok{, }\DataTypeTok{width  =} \DecValTok{1}\NormalTok{, }\DataTypeTok{color =}\NormalTok{ pal[}\KeywordTok{max}\NormalTok{(n_lev)]) }\OperatorTok{+}
\StringTok{  }\KeywordTok{scale_alpha_manual}\NormalTok{(}\DataTypeTok{values =} \KeywordTok{c}\NormalTok{(}\KeywordTok{rep}\NormalTok{(}\DecValTok{1}\NormalTok{, (n_lev}\OperatorTok{-}\DecValTok{1}\NormalTok{)), }\DecValTok{0}\NormalTok{)) }\OperatorTok{+}\StringTok{ }
\StringTok{  }\KeywordTok{scale_color_manual}\NormalTok{(}\DataTypeTok{values =}\NormalTok{ pal[}\KeywordTok{seq}\NormalTok{(}\DecValTok{1}\NormalTok{, n_lev)]) }\OperatorTok{+}
\StringTok{  }\CommentTok{# above we set the line between rec fisheries transparent}
\StringTok{  }\KeywordTok{labs}\NormalTok{(}\DataTypeTok{x =} \StringTok{""}\NormalTok{, }\DataTypeTok{y =}\NormalTok{ y_axis) }\OperatorTok{+}
\StringTok{  }\KeywordTok{ggtitle}\NormalTok{(title) }\OperatorTok{+}
\StringTok{  }\KeywordTok{guides}\NormalTok{(}\DataTypeTok{fill  =} \OtherTok{FALSE}\NormalTok{,}
         \DataTypeTok{alpha =} \OtherTok{FALSE}\NormalTok{,}
         \DataTypeTok{color =} \KeywordTok{guide_legend}\NormalTok{(}\DataTypeTok{nrow =} \DecValTok{3}\NormalTok{, }
                              \DataTypeTok{title =} \StringTok{""}\NormalTok{,}
                              \DataTypeTok{override.aes =} \KeywordTok{list}\NormalTok{(}\DataTypeTok{size =} \FloatTok{1.3}\NormalTok{, }
                                                  \DataTypeTok{color =}\NormalTok{ pal[}\KeywordTok{seq}\NormalTok{(}\DecValTok{1}\NormalTok{, n_lev)]),}
                              \DataTypeTok{keywidth =} \FloatTok{0.3}\NormalTok{,}
                              \DataTypeTok{keyheight =} \FloatTok{0.1}\NormalTok{,}
                              \DataTypeTok{default.unit =} \StringTok{"inch"}\NormalTok{)) }\OperatorTok{+}
\StringTok{  }\KeywordTok{scale_x_continuous}\NormalTok{(}\DataTypeTok{expand =} \KeywordTok{c}\NormalTok{(}\DecValTok{0}\NormalTok{, }\DecValTok{0}\NormalTok{), }\DataTypeTok{breaks =}\NormalTok{ scales}\OperatorTok{::}\KeywordTok{pretty_breaks}\NormalTok{(}\DataTypeTok{n =} \DecValTok{6}\NormalTok{)) }\OperatorTok{+}
\StringTok{  }\KeywordTok{scale_y_continuous}\NormalTok{(}\DataTypeTok{expand =} \KeywordTok{c}\NormalTok{(}\DecValTok{0}\NormalTok{, }\DecValTok{0}\NormalTok{), }\DataTypeTok{breaks =}\NormalTok{ scales}\OperatorTok{::}\KeywordTok{pretty_breaks}\NormalTok{(}\DataTypeTok{n =} \DecValTok{5}\NormalTok{)) }\OperatorTok{+}
\StringTok{  }\KeywordTok{theme_rom}\NormalTok{()}
\end{Highlighting}
\end{Shaded}

\begin{figure}

{\centering \includegraphics{Making_figures_for_RoM_in_R_files/figure-latex/plot 2-1} 

}

\caption{Gädda in the Great Lakes (RoM 2018) - example of a multiple-series plot using R}\label{fig:plot 2}
\end{figure}

Now save the file (to your working directory)

\begin{Shaded}
\begin{Highlighting}[]
\KeywordTok{ggsave}\NormalTok{(}\StringTok{"Fig_2.tiff"}\NormalTok{, }\DataTypeTok{plot =}\NormalTok{ p2, }\DataTypeTok{dpi =} \DecValTok{300}\NormalTok{, }\DataTypeTok{width =} \DecValTok{8}\NormalTok{, }\DataTypeTok{height =} \DecValTok{8}\NormalTok{, }\DataTypeTok{units =} \StringTok{"cm"}\NormalTok{)}
\end{Highlighting}
\end{Shaded}

Let's say that you have 3 different areas and no special series and no
error bars. Then you simply need to hash tag the code the plots those
features. We subset the full data to select only the total and lake
Mälaren as an example:

\begin{Shaded}
\begin{Highlighting}[]
\NormalTok{dat2 <-}\StringTok{ }\NormalTok{dat }\OperatorTok\StringTok{ }
\StringTok{  }\KeywordTok{select}\NormalTok{(År, Biomassa, Område) }\OperatorTok\StringTok{ }
\StringTok{  }\KeywordTok{filter}\NormalTok{(Område }\OperatorTok\StringTok{ }\KeywordTok{c}\NormalTok{(}\StringTok{"Stora sjöarna"}\NormalTok{, }\StringTok{"Mälaren"}\NormalTok{))}
\end{Highlighting}
\end{Shaded}

We now need to update the number of levels in the data:

\begin{Shaded}
\begin{Highlighting}[]
\NormalTok{n_lev <-}\StringTok{ }\KeywordTok{length}\NormalTok{(}\KeywordTok{unique}\NormalTok{(dat2}\OperatorTok{$}\NormalTok{Område)) }
\end{Highlighting}
\end{Shaded}

Repeat the general Fig. 2 plot, but remove the points and error bars:

\begin{Shaded}
\begin{Highlighting}[]
\NormalTok{p2a <-}\StringTok{ }\KeywordTok{ggplot}\NormalTok{(dat2, }\KeywordTok{aes}\NormalTok{(År, Biomassa, }\DataTypeTok{color =}\NormalTok{ Område)) }\OperatorTok{+}
\StringTok{  }\KeywordTok{geom_bar}\NormalTok{(}\DataTypeTok{data =} \KeywordTok{subset}\NormalTok{(dat2, Område }\OperatorTok{==}\StringTok{ }\NormalTok{main_series), }
           \KeywordTok{aes}\NormalTok{(}\DataTypeTok{x =}\NormalTok{ År, }\DataTypeTok{y =}\NormalTok{ Biomassa), }\DataTypeTok{stat =} \StringTok{"identity"}\NormalTok{, }\DataTypeTok{color =}\NormalTok{ pal[}\DecValTok{1}\NormalTok{], }\DataTypeTok{fill =}\NormalTok{ pal[}\DecValTok{1}\NormalTok{], }
           \DataTypeTok{width =} \FloatTok{0.6}\NormalTok{) }\OperatorTok{+}
\StringTok{  }\KeywordTok{geom_line}\NormalTok{(}\DataTypeTok{data =}\NormalTok{ dat2, }\KeywordTok{aes}\NormalTok{(År, Biomassa, }\DataTypeTok{color =}\NormalTok{ Område, }\DataTypeTok{alpha =}\NormalTok{ Område), }
            \DataTypeTok{size =} \DecValTok{1}\NormalTok{) }\OperatorTok{+}\StringTok{ }
\StringTok{  }\CommentTok{#geom_point(data = subset(dat, Område == special_series), # here we define our special series}
\StringTok{  }\CommentTok{#           aes(År, Biomassa, fill = Område), size = 2, color = pal[max(n_lev)]) + }
\StringTok{  }\CommentTok{#geom_errorbar(data = subset(dat, Område == special_series), }
\StringTok{  }\CommentTok{#              aes(x = År, ymin = rec_minu, ymax = rec_plus), }
\StringTok{  }\CommentTok{#              show.legend = FALSE, width  = 1, color = pal[max(n_lev)]) +}
\StringTok{  }\CommentTok{#scale_alpha_manual(values = c(rep(1, (n_lev-1)), 0)) + }
\StringTok{  }\KeywordTok{scale_color_manual}\NormalTok{(}\DataTypeTok{values =}\NormalTok{ pal[}\KeywordTok{seq}\NormalTok{(}\DecValTok{1}\NormalTok{, n_lev)]) }\OperatorTok{+}
\StringTok{  }\KeywordTok{labs}\NormalTok{(}\DataTypeTok{x =} \StringTok{""}\NormalTok{, }\DataTypeTok{y =}\NormalTok{ y_axis) }\OperatorTok{+}
\StringTok{  }\KeywordTok{ggtitle}\NormalTok{(title) }\OperatorTok{+}
\StringTok{  }\KeywordTok{guides}\NormalTok{(}\DataTypeTok{fill  =} \OtherTok{FALSE}\NormalTok{,}
         \DataTypeTok{alpha =} \OtherTok{FALSE}\NormalTok{,}
         \DataTypeTok{color =} \KeywordTok{guide_legend}\NormalTok{(}\DataTypeTok{nrow =} \DecValTok{3}\NormalTok{, }
                              \DataTypeTok{title =} \StringTok{""}\NormalTok{,}
                              \DataTypeTok{override.aes =} \KeywordTok{list}\NormalTok{(}\DataTypeTok{size =} \FloatTok{1.3}\NormalTok{, }
                                                  \DataTypeTok{color =}\NormalTok{ pal[}\KeywordTok{seq}\NormalTok{(}\DecValTok{1}\NormalTok{, n_lev)]),}
                              \DataTypeTok{keywidth =} \FloatTok{0.3}\NormalTok{,}
                              \DataTypeTok{keyheight =} \FloatTok{0.1}\NormalTok{,}
                              \DataTypeTok{default.unit =} \StringTok{"inch"}\NormalTok{)) }\OperatorTok{+}
\StringTok{  }\KeywordTok{scale_x_continuous}\NormalTok{(}\DataTypeTok{expand =} \KeywordTok{c}\NormalTok{(}\DecValTok{0}\NormalTok{, }\DecValTok{0}\NormalTok{), }\DataTypeTok{breaks =}\NormalTok{ scales}\OperatorTok{::}\KeywordTok{pretty_breaks}\NormalTok{(}\DataTypeTok{n =} \DecValTok{6}\NormalTok{)) }\OperatorTok{+}
\StringTok{  }\KeywordTok{scale_y_continuous}\NormalTok{(}\DataTypeTok{expand =} \KeywordTok{c}\NormalTok{(}\DecValTok{0}\NormalTok{, }\DecValTok{0}\NormalTok{), }\DataTypeTok{breaks =}\NormalTok{ scales}\OperatorTok{::}\KeywordTok{pretty_breaks}\NormalTok{(}\DataTypeTok{n =} \DecValTok{5}\NormalTok{)) }\OperatorTok{+}
\StringTok{  }\KeywordTok{theme_rom}\NormalTok{()}
\end{Highlighting}
\end{Shaded}

\begin{figure}

{\centering \includegraphics{Making_figures_for_RoM_in_R_files/figure-latex/plot 2a-1} 

}

\caption{Gädda in the Great Lakes (RoM 2018) - example of a multiple-series plot using R}\label{fig:plot 2a}
\end{figure}

\subsubsection{To work on:}\label{to-work-on}

\begin{itemize}
\tightlist
\item
  secondary y-axis that is not a one-to-one transformation of the
  primary axes. Possible?
\end{itemize}


\end{document}
